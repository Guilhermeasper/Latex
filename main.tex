\documentclass{article}
\usepackage[utf8]{inputenc}

\title{League of Legends}
\author{Guilherme Afonso}
\date{Outubro 2018}

\usepackage{natbib}
\usepackage{graphicx}
\usepackage[brazil]{babel}
\usepackage[utf8]{inputenc}
\usepackage[T1]{fontenc}
\usepackage[normalem]{ulem}
\useunder{\uline}{\ul}{}

\begin{document}

\maketitle

\section{Introdução}
League of Legends\cite{wiki:League_of_Legends} (abreviado como LoL) é um jogo eletrônico do gênero multiplayer online battle arena, desenvolvido e publicado pela Riot Games para Microsoft Windows e Mac OS X. É um jogo gratuito para jogar e inspirado no modo Defense of the Ancients (abreviado como DotA\cite{wiki:Dota}) de Warcraft III: The Frozen Throne.

\begin{figure}[h!]
\centering
\includegraphics[scale=0.2]{league}
\caption{Imagem promocional do popular game on-line League of Legends}
\label{fig:league}
\end{figure}

\section{Pontos positivos e negativos}
Prós:
\begin{itemize}
   \item O jogo é grátis
   \item Diversidade enorme de campeões e habilidades
   \item Bem otimizado para máquinas com baixo processamento
   \item Sempre há novas atualizações
   \item Modos rotativos que quebram a monotonia do jogo
   \item Sistema de recompensas grátis
 \end{itemize}
 \vspace{0.5cm}
 \noindent
Contras:
\begin{itemize}
   \item Muito campões com mecânica parecida
   \item Desenvolvedora não aceita feedback da comunidade
   \item Atualizações grandes em intervalos pequenos de tempo
   \item Comunidade extremamente tóxica
   \item Cenário competitivo fraco
\end{itemize}
\section{Comparativo com DotA 2}

\begin{table}[h!]
\begin{tabular}{|c|c|c|}
\hline
{\ul \textit{}}               & \textbf{LoL}                                                         & \textit{\textbf{DotA 2\cite{wiki:Dota_2}}}                                              \\ \hline
\textbf{Gráficos}             & Muito bons                                                           & Ótimos                                                              \\ \hline
\textbf{Duração das partidas} & 30 min                                                               & 50 min                                                              \\ \hline
\textbf{Campeonatos}          & \begin{tabular}[c]{@{}c@{}}Muitos/Pequenas\\ premiações\end{tabular} & \begin{tabular}[c]{@{}c@{}}Poucos/Grandes\\ premiações\end{tabular} \\ \hline
\textbf{Dificuldade}          & Muito fácil                                                          & Muito díficil                                                       \\ \hline
\textbf{Mapas}                & 4 Mapas diferentes                                                   & Somente um mapa                                                     \\ \hline
\end{tabular}
\end{table}

\bibliographystyle{plain}
\bibliography{references}
\end{document}
